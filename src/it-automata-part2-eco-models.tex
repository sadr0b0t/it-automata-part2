% https://habr.com/ru/companies/ruvds/articles/574352/
% https://guides.nyu.edu/LaTeX/sample-document
% https://ru.stackoverflow.com/questions/222769/latex-как-быть-с-русским-текстом/757767#757767

\documentclass{article}
% без этой строчки (модуль cmap) не будет работать поиск внутри документа и копирование русского текста:
\usepackage{cmap}
\usepackage[utf8]{inputenc}
\usepackage[english,russian]{babel}

\title{Экономические модели распространения цифровых технологий}
\author{А.~Е.~Моисеев}
\date{май 2024}
%\date \today

\begin{document}

\maketitle

Распорядители ресурсов, принимая решение о целесообразности реализации нового технологического продукта, могут опираться на такие критерии, как оценка объема целевого рынка в денежном выражении, количеством потенциальных потребителей и т.~п. Такой подход позволяет получить инвесторам некоторые количественные оценки для верхних и промежуточных планок возможных финансовых результатов проекта. Вместе с тем, он опирается на наличие существующего рынка и его объем как на данность, при этом не даёт прямого ответа на вопрос, почему этот рынок изначально возник, каковы объективные стимулы потребителей покупать продукты, которые на нем представлены.

Внедрение технологий, основанных на автоматических цифровых вычислениях, позволяет потребителям повышать производительность труда — создавать больше стоимости на производстве или сокращать непроизводственные издержки, таким образом достигать экономического эффекта \cite{ecoEffects}. Стремление получить экономический эффект — объективный стимул покупки и внедрения технологического продукта. С точки зрения потребителя, результат сравнения цены продукта и потенциального экономического эффекта, который от его внедрения можно ожидать, является объективным критерием принятия решения о покупке. С точки зрения производителя продукта, совокупный экономический эффект, который может быть достигнут всеми потенциальными потребителями продукта на рынке, является верхней границей ресурсов рынка, на которые он может претендовать.

Экономический эффект от внедрения ПО достигается при исполнении программных алгоритмов на аппаратной платформе. Чем больше действующих установок программного продукта развернуто и эксплуатируется по назначению, тем больший суммарный экономический эффект они произведут. Программное обеспечение может распространяться как «чистое» ПО, когда наличие вычислительной техники у потребителя подразумевается как исходная предпосылка распространения программного модуля, или в составе аппаратно-программной платформы. В случае с вариантом аппаратно-программной платформы количество копий программы привязано к количеству произведенных и проданных устройств. В случае с «чистым» ПО возможности распространения программного продукта ограничены распространенностью целевой аппаратной платформы.

% команда \chapter недоступна в documentclass article
%\chapter{Аппаратно-программные платформы}
\section*{Аппаратно-программные платформы}

Аппаратно-программные платформы: бытовая техника с встроенными микроконтроллерами, гаджеты, роботы, станки с ЧПУ и т.~п. Программное обеспечение наделяет аппаратную часть необходимыми потребительными свойствами: микроволновая печь, стиральная машина с цифровым управлением или смартфон не сможет работать без загруженного на него базового пакета ПО. Аппаратная платформа без загруженного базового пакета программного обеспечения — незаконченное устройство, полуфабрикат. Аппаратная платформа с загруженным пакетом программного обеспечения — законченное специализированное устройство. Потребитель покупает законченное устройство, включающее программную прошивку, но не отдельный программный пакет. Экономический эффект в таком случае следует рассчитывать от внедрения готового устройства, а не отдельных программных модулей.

Разработка пакета программного обеспечения, предназначенного специально для аппаратно-программного устройства, — этап инженерно-конструкторских работ наряду с такими задачами, как выбор и компоновка аппаратных компонент, разработка электронной схемы печатных плат, проектирование корпуса устройства и т.~п. Развертывание программного обеспечения (обычно уже в форме двоичной прошивки) на аппаратной платформе — акт производства наряду с изготовлением деталей и сборкой устройства. 

Разработка ПО происходит один раз для партии из любого количества устройств до этапа серийного производства, производится инженерами-программистами, тестерами, прочими членами команды разработки. Развертывание прошивки происходит для каждого производимого устройства на этапе серийного производства, производится рабочими производственной линии — операторами специального оборудования, а не программистами.

Стоимость инженерно-конструкторских работ, включая стоимость разработки программного обеспечения, — фиксированная величина, не зависящая от объема партии производства спроектированных устройств. Абсолютная суммарная прибыль от продажи партии произведенных устройств будет тем больше, чем больше произведено устройств. Размер партии, запланированной к производству, должен быть достаточно большим для того, чтобы прибыль от её продажи покрыла фиксированные издержки на инженерное проектирование, в том числе на разработку программного обеспечения. В том случае, если запланированная партия не достаточно велика настолько, чтобы расчетная прибыль превысила стоимость разработки, запуск проекта не имеет экономического смысла.

В некоторых случаях производитель имеет возможность выпустить обновления для базовой программной прошивки и распространить их через такие механизмы, как OTA (over the air update — обновление «по воздуху») для смартфонов, или бесплатные обновления безопасности на сайте производителя. В таком случае издержки на разработку программного кода обновления переносятся на период, следующий за стадией производства и продажи устройств. Однако и в этом случае стоимость разработки каждого обновления остаётся фиксированной величиной и сама по себе не зависит от количества выпущенных ранее и проданных устройств. Крупнейшие производители смартфонов на платформе Гугл Андроид выпускают от двух до семи значительных обновлений системы за срок от двух до семи лет после релиза устройства \cite{phoneUpdates}, при этом больше внимания уделяют флагманским моделям и меньше — бюджетным.

\section*{Издержки на разработку ПО}

В современной индустрии разработки программного обеспечения сложно представить проект, который был бы написан силами одной команды абсолютно «с нуля». Так или иначе любая разработка задействует ранее написанный код — стандартную библиотеку языка, драйверы устройств, операционную систему, прочие сторонние библиотеки подпрограмм. Таким образом, любая программная разработка состоит из двух больших частей: новый код, написанный специально для реализации проекта, и старый код, полученный извне, который вызывается из нового кода в виде подпрограмм или работает параллельно с ним на одной аппаратной платформе.

Новый внутренний по отношению к проекту код обычно заточен под конкретный проект и границы его внедрения очерчены размером партии разработанного устройства. Старый внешний по отношению к проекту код обычно в значительной степени универсален, границы его применимости не очерчены размером партии конкретного устройства — одна и та же программная библиотека применяется в множестве проектов и работает в составе прошивки на множестве разнообразных устройств.

Универсальные программные модули обычно являются разработкой отдельных команд и организаций. Распространяются в форме коммерческого программного обеспечения, например, по модели продажи лицензий, или в форме свободного программного обеспечения — под лицензией, подразумевающей свободное использование, изменение и распространение программного продукта без обязательных денежных отчислений за право использовать кодовую базу в том числе в коммерческих проектах.

Разработчик коммерческого программного модуля так же, как и непосредственный разработчик устройства, авансирует собственные средства на первоначальную разработку до того, как первая версия модуля войдет в состав прошивки реального устройства. Компенсация средств, потраченных на разработку стороннего программного модуля, в конечном итоге также будет производиться из прибыли от продажи устройств: производитель устройства сначала передаст часть собственных средств разработчику в форме платы за лицензию на программное обеспечение, а потом компенсирует собственные расходы из продажной прибыли. Однако количество внедрений универсального программного модуля не будет ограничено размером партии конкретного устройства, компенсировать исходную разработку будет не один производитель, а множество игроков. При том, что стоимость исходной разработки модуля остаётся фиксированной, она теперь распределена между большим количеством производителей. Покупка лицензии на программный модуль может обойтись каждому из производителей гораздо дешевле, чем затраты в ситуации, когда бы они разрабатывали аналогичные модули каждый для себя собственными силами.

Для того, чтобы производитель аппаратного обеспечения имел объективный экономический стимул купить лицензию на сторонний коммерческий программный модуль, разработчику программного модуля достаточно выставить такую цену лицензии, которая будет ниже стоимости разработки аналогичного программного модуля силами разработчика аппаратного обеспечения. Чем большее количество устройств от разных производителей сможет использовать коммерческий программный модуль, тем меньшая цена может быть выставлена на продажу лицензии, тем больше стимула разработчику устройства лицензировать чужой код, а не разрабатывать аналогичный модуль самостоятельно.

Таким образом, наличие сторонних модулей позволяет понизить порог выхода на рынок производителям новых устройств, понижая минимальный размер партии, необходимый для оправдания расходов на разработку программной прошивки.

После того, как все исходные расходы на разработку программного модуля компенсированы из лицензионных отчислений, денежный поток от последующих продаж новых лицензий составит для разработчика коммерческого модуля чистый доход. Он сможет их тратить на поддержку программного продукта (выпуск патчей и обновлений — обычно это незначительные расходы по сравнению со стоимостью исходной разработки), на операционные издержки, а также авансировать на разработку новых проектов (в том числе новых версий существующего продукта), стоимость разработки которых будет компенсирована по такой же модели из будущих внедрений.

Расходы на разработку свободного программного обеспечения берет на себя общество в целом или отдельные коммерческие игроки, бизнес которых не базируется на модели прямой продажи лицензий на программное обеспечение. Разработчики устройств могут свободно использовать такой код и не платить за него прямых лицензионных отчислений, порог выхода на рынок таким образом для них опускается еще ниже.

Однако любой программный продукт, как коммерческий, так и с открытым исходным кодом, может требовать дополнительный труд по доработке и настройке для задач конкретного проекта. Эти задачи решит внутренняя команда разработчиков проекта, поэтому применение сторонних программных модулей может значительно понизить стоимость разработки программного обеспечения для устройства, но в большинстве случаев не сможет обратить её в ноль.

\section*{Компенсация средств, авансированных на разработку программного обеспечения: бизнес-модели распространения и продажи ПО}

Создание нового программного продукта занимает продолжительное время, в течение этого времени он не может быть использован по назначению. Издержки на первоначальную разработку покрываются авансом из первоначальных инвестиций — накопленного капитала. На первых этапах (стадия разработки) они значительны, после того, как продукт разработан и стабилизирован (стадия эксплуатации), их можно радикально сократить.

На этапе разработки инженеры-программисты получают средства инвестора — эти средства составят стоимость разработки. На этапе эксплуатации и продаж инвестор возвращает себе средства, потраченные на инженеров-разработчиков. Средства, полученные от продаж, компания-разработчик может реинвестировать\footnote{«A significant investment is made to develop products. And survival is dependent on our partners’ ability to distribute and license products so that a reinvestment can be made into research and development to create new and better software products to meet consumers’ needs.» \cite{microsoftPiracyReinvest}} в создание новой версии программного продукта с тем же названием и той же командой так, что со стороны это будет выглядеть, как непрерывный процесс разработки одного и того же программного продукта. Однако, в действительности, на стадии эксплуатации, приносящей доход, будет находиться продукт старой версии, а продукт новой не выпущенной в эксплуатацию версии будет находиться на стадии разработки, на которой средства расходуются на команду разработки, но не возвращаются до тех пор, пока новая версия продукта не перейдет в стадию эксплуатации.

Объективный стимул внедрения программного продукта его потребителем — желание получить новую стоимость (или необходимость тратить не больше, чем остальные), т.~е. реализовать экономический эффект. Часть этой новой или высвободившейся стоимости — источник средств, из которых потребитель сможет сначала компенсировать собственные расходы на покупку и внедрение программного продукта, которые, в свою очередь, ранее уже компенсировали разработчику средства, потраченные на разработку программного продукта. Т.~е., в конечном итоге, первоначальные расходы, направленные инвестором на разработку программного продукта, будут компенсированы частью суммарного экономического эффекта от внедрений этого программного продукта среди всех его потребителей во всём обществе.

Теперь задача разработчика — организовать канал движения средств, через который он эту стоимость сможет получить от внедряющей программный продукт стороны. Вопросы на этом этапе — целевая бизнес-модель и ценообразование.

\section*{Внутрикорпоративная разработка}

Результат разработки внедряется и используется исключительно внутри компании, не используется за её пределами. Источник финансирования — свободные собственные средства компании — накопленный капитал. Разработка может быть проведена внутренним отделом ИТ или заказана на стороне с условием передачи исключительных прав на использование продукта внутри компании-заказчика. Экономический эффект от внедрения разработанного продукта окупит первоначальные издержки на разработку, далее компания продолжит работу с новой повышенной производительностью труда. Компания должна быть достаточно крупной для того, чтобы общий экономический эффект от внедрения разработанного программного пакета окупил издержки на его разработку. Чем больше компания, тем больший масштаб внутренних разработок она может себе позволить.

Некоторые платформы автоматизации бизнеса такие, как 1С, позволяют создавать персонализированные продукты, при этом возможности базовой платформы снижают стоимость разработки индивидуальных конфигураций. Базовая платформа выступает полуфабрикатом, в производственный процесс внедряется построенная на его базе индивидуальная конфигурация. Стоимость разработки индивидуальной конфигурации компенсируется при этом целиком из экономического эффекта единственного потребителя, стоимость разработки базовой платформы компенсируется по модели коммерциализации тиражируемого массово программного продукта.

Дополнительно: Гугл, Фейсбук выкладывают внутренние разработки в открытый доступ, таким образом экономический эффект от внедрения распространяется на масштаб экономики (например: Apache Cassandra, ReactJS), Роснефть собиралась коммерциализировать внутренние разработки на открытом рынке (todo: ссылка из прошлой статьи).

\section*{«Продажа» лицензий и подписка}

Хотя в документах учёта продажа и покупка лицензионного ПО проводится по особым процедурам \cite{accountingIAS, accountingNKRF257, accountingNKRF264}, к форме сделки лицензирования, т.~е. получения неисключительного права на внедрение и использование программного продукта, часто применяют метафору купли-продажи материального объекта, по крайней мере проводят между процедурами «покупки» лицензии на программный продукт и покупки материального продукта параллели \cite{driscollOpenLetter}. Механизм покупки лицензии на ПО даёт возможность разработчику ПО получить компенсацию от потребителя ПО в форме платежа по договору. Таким образом, продав известное количество лицензий, разработчик компенсирует первоначальные издержки на разработку — компенсирует инвестиции, все последующие продажи — чистый доход. Покупатель, получив право и возможность запускать ПО на своём аппаратном обеспечении, внедряет его в процесс производства, повышает производительность труда, компенсирует издержки на покупку лицензии из экономического эффекта автоматизации — сокращения издержек или наращивания производства дополнительного продукта, которые он получил следствием внедрения ПО.

Модель продажи лицензий на первый взгляд интуитивно понятна за счет метафорической аналогии с материальными товарами. Однако природа ПО как интеллектуального продукта определяет стратегию присутствия на рынке и ценообразования ИТ-компаний, отличную от стратегии производителей материальных благ. Цена единицы материального товара определяется издержками на его производство и транспортировку с поправкой на колебания цен под действием спроса и предложения и прочих экономических факторов, производство каждой новой партии одного и того же продукта требует понести известное количество затрат, процесс воспроизводства материальных продуктов цикличен. Стоимость разработки программного продукта конечна, программный продукт не воспроизводится заново для каждого нового внедрения, а создаётся один раз, стоимость создания каждой новой копии программного продута пренебрежимо мала \cite{introToDigital}. Для того, чтобы рассчитать возможную цену лицензии на копию программного продукта, следует принять во внимание интерес разработчика, с одной стороны, и интерес потребителя — с другой.

Со стороны потребителя максимальная цена, которую целесообразно платить за копию ПО, определяется потенциальным экономическим эффектом от его внедрения. Если цена лицензии на копию ПО окажется выше сэкономленных в результате его внедрения средств, то такая покупка принесет итоговый убыток. Если цена лицензии окажется равной экономическому эффекту, покупатель ничего не выиграет, но и не проиграет от покупки такого ПО. Если цена лицензии окажется ниже экономического эффекта автоматизации, покупатель окажется в плюсе от покупки и внедрения ПО, и это будет являться для него объективным экономическим стимулом совершить покупку.

Т.~к. стоимость создания каждой новой копии программного продукта для разработчика пренебрежимо мала, он может позволить себе выставить любой вариант цены единичной лицензии. Разработчик, тем не менее, ограничен издержками на первоначальную разработку программного продукта, а также некоторым количеством фиксированных (не зависящих от количества «производимых» копий) регулярных операционных издержек. Все проданные лицензии в сумме должны, как минимум, компенсировать эти расходы. Цена одной лицензии может быть рассчитана как общие издержки на разработку ПО, делёные на количество лицензий, которые можно потенциально продать на выбранном сегменте рынка в оговоренный период.

Если рассчитанная по такой формуле цена окажется выше, чем средний ожидаемый экономический эффект автоматизации для большинства потенциальных потребителей ПО, то необходимое количество лицензий не будет продано, разработка изначально не будет иметь шансов окупиться. Если ниже, то разработка проекта имеет смысл.

Покупатели программного продукта — организации или физические лица. После того, как значительная часть всех пользователей купила лицензию и установила ПО, новых продаж не будет. Нужно повторять цикл — выпускать новую версию ПО и продавать новые лицензии \cite{windowsXPLegacy}. В отличие от материального продукта, программное обеспечение со временем не изнашивается, но оно может морально устареть в том случае, если на рынке появится программный продукт, экономический эффект от внедрения которого в достаточной степени превзойдет экономический эффект от эксплуатации рассматриваемого пакета. Для того, чтобы существующие пользователи программного продукта имели объективный стимул купить его новую версию, экономический эффект от её внедрения должен превышать издержки перехода с предыдущей версии. Таким образом, чтобы обеспечить необходимое количество продаж среди старой базы потребителей, новую версию следует выпускать с достаточным количеством изменений. Или ограничить время жизни лицензии, вместо разовой «покупки» получить регулярные платежи за один и тот же продукт — использовать модель подписки \cite{autodeskSubscription}.

При нулевой стоимости копирования продукта компании-разработчики ПО имеют возможность вести гибкую ценовую политику — выставлять разные цены на один и тот же продукт для различных категорий потребителей, в т.~ч. устанавливать нулевые цены в отдельных сегментах, насыщать новый рынок своим ПО с нулевыми затратами (все затраты на разработку уже понесены), вытесняя конкурентов. К примеру, в 2009-м году компания Microsoft объявила, что планирует «инвестировать» в Россию 10 млрд. руб. \cite{microsoftInvestsRF}. В действительности в рамках программы планировалось реализовать такие мероприятия, как «запуск в России программы обучения базовым компьютерным навыкам», а также предоставить «более 1000 российских компаний» бесплатно «широкий набор программных продуктов Microsoft», студентам и школьникам — «бесплатный доступ к современным инструментам разработки и дизайна Microsoft». Таким образом, под значительной долей объявленных инвестиций в этом случае подразумевалось не выделение новых средств в денежной форме или форме аппаратного обеспечения, а развертывание новых легальных или легализация существующих установок ПО, которые позволили укрепить позиции компании на рынке и были использованы в рекламной кампании, не принеся при этом никаких дополнительных расходов.

\section*{SaaS: «ПО как сервис»}

Модель SaaS — «Software as a service», «ПО как сервис», «ПО как услуга» подразумевает схему работы с программным обеспечением, запущенным на аппаратном обеспечении продавца сервиса (в т.~н. «облаке»\footnote{Rob Joyce, former chief of the Tailored Access Operations at the U.S. National Security Agency (NSA): «cloud computing is really just a fancy name for someone else’s computer.» \cite{cloudIsAFancyName}}), покупатель получает к нему доступ через интернет через браузерный веб-интерфейс.

По форме модель SaaS часто преподносят как решение «проблемы» бессрочных лицензий на оффлайн-ПО\footnote{«Yet the total costs of SaaS need to be calculated, he says. “Think of renting software. That’s what software-as-a-service is in terms is pricing models – you’re subscribing to that. So you have the whole lease vs. buy discussion with the car dealer. If you’re going to drive this thing until the doors fall off, you should buy it. If you get a new car every three years, you should lease it.”» \cite{microsoftSaasStrategy}}, приводящих к исчерпанию рынка установок и ведущих к необходимости регулярно выпускать новые версии ПО с большим количеством изменений. С моделью SaaS доступ к облачной инфраструктуре всегда покупается на время, одна и та же база пользователей обеспечивает постоянный денежный поток независимо от того, насколько часто и в какой мере обновляется программное обеспечение. В отличие от модели срочных лицензий, т.~е. оффлайн-подписки, после завершения срока подписки на облачный сервис онлайн-доступ технически легко отключить\footnote{«It makes perfect sense for the vendors as it ensures that they are fully paid for the use of their software, as pirating within cloud-based subscription licensing hasn’t been cracked yet.» \cite{microsoftMoveToSubscriptionAndSaas}}.

С точки зрения пользователя для получения экономического эффекта от применения ПО по прямому назначению нет принципиальной разницы, запущенно ли ПО локально или работает по модели SaaS. В отдельных сценариях применение сервиса SaaS может принести дополнительную экономию на издержках содержания ИТ-инфраструктуры. Продавец услуги не распространяет ПО в виде копий, оно запущено и работает на его аппаратном обеспечении. Все вместе взятые потребители услуги должны компенсировать ему исходную стоимость аппаратной платформы и текущие расходы по поддержанию её в рабочем состоянии — электричество и присоединенная стоимость труда обслуживающего персонала, в том числе системных администраторов. Таким образом, потребитель покупает не программное обеспечение, а покупает аппаратное обеспечение, энергию и труд по частям. Программное обеспечение здесь наделяет необходимыми потребительными качествами универсальное аппаратное обеспечение — превращает его в специализированный автомат, создавая условия для его покупки и полезной эксплуатации. Сервис SaaS — это специализированный аппаратно-программный комплекс, потребляемый коллективно через канал связи интернет. Подписка на сервис SaaS — договор срочной аренды аппаратного обеспечения. 

Суммарная необходимая цена подписки за выбранный период может быть рассчитана по общей схеме: амортизация аппаратного обеспечения в пересчете на выбранный период, стоимость энергии, стоимость канала связи, стоимость помещения, присоединенная стоимость, создаваемая трудом обслуживающего персонала — системных администраторов в датацентре. Расчетная средняя цена подписки для одного клиента в таком случае будет равна суммарной необходимой цене подписки, делёной на максимальное количество клиентов, которое может обслужить данный сервис. Чем больше клиентов обслуживает сервис, тем более низкая цена может быть выставлена для средней единичной подписки. Максимальное количество клиентов сервиса, а значит минимальная цена подписки, будут ограничены, во-первых, возможностями аппаратного обеспечения, которое в рамках работы сервиса обслуживает всех клиентов на выбранном периоде подписки \cite{erpSaas}. Во-вторых, объемом рынка — общим количеством потенциальных клиентов, которые могут быть заинтересованы в услуге. Отдельные опции аппаратного обеспечения, к примеру, дополнительное дисковое пространство, могут быть предложены клиентам для индивидуальной оплаты без необходимости усреднять цену по всей клиентской базе, или же финальная цена будет рассчитана динамически в зависимости от фактически потребленных ресурсов в конце каждого платежного периода \cite{sellingServices}.

С точки зрения потребителя услуги, если цена подписки окажется ниже экономического эффекта автоматизации, покупатель окажется в плюсе от подписки на сервис и внедрения его в свой производственный процесс, это будет являться для него объективным экономическим стимулом совершить покупку. Если для большинства потенциальных покупателей выгода от подписки окажется меньше её цены, запущенный сервис рискует остаться без потребителей.

Стоимость разработки ПО для облачного сервиса будет постепенно компенсирована из прибыли от продажи аппаратного обеспечения сервиса. Стоимость первоначального развертывания ПО на аппаратной платформе присоединяется к первоначальной стоимости аппаратного обеспечения и включается в продажную цену сервиса. Аппаратная инфраструктура не обязательно принадлежит непосредственно провайдеру сервиса, она может быть арендована им у третей стороны \cite{dropboxBuildsOwnInfra}, дополнена необходимым программным обеспечением и в таком виде перепродана конечному потребителю.

Затратна предварительная разработка ПО. Для аппаратно-программной платформы значительную часть кодовой базы можно взять из проектов с открытым исходным кодом или лицензировать у сторонних разработчиков. Некоторые проекты представляют открытые реализации программного обеспечения, готовые для развертывания и запуска сервиса целиком (например: WordPress, NextCloud, OnlyOffice), однако многие крупные игроки на рынке SaaS предпочитают использовать собственные закрытые разработки, работающие исключительно на их сервисах (например: DropBox, Яндекс Диск, MS Office 365).

\section*{Продажа поддержки программного обеспечения}

Эксплуатация программного обеспечения на собственной аппаратно-программной инфраструктуре требует вовлечения труда технических специалистов — системных администраторов, технических консультантов и т.~п. Для потребителя ПО это выливается в известное количество эксплуатационных издержек, которые он несет на регулярной основе. В ряде ситуаций может быть целесообразнее часть этих работ вынести во внешнюю организацию: если потребность в работе специалиста по тому или иному программному пакету возникает нерегулярно, если настройка и эксплуатация программного продукта требует специальных навыков, не распространенных среди доступных на рынке труда специалистов, если диагностика возможных проблем с программным пакетом требует развертывания специальной инфраструктуры разработки или доступ к закрытой кодовой базе и т.~п.

Производители программного обеспечения для автоматизации бизнеса кроме, собственно, программного пакета, могут предлагать клиентам дополнительные услуги с оплатой по подписке. Такая подписка обычно включает доступ к закрытым обновлениям ПО (критические обновления безопасности при этом могут публиковаться бесплатно), а также к службе поддержки \cite{redhatSubscriptionModel}. Платные обновления — это форма коммерческого распространения программного обеспечения, по существу аналогичная модели продажи лицензий. Поддержка представляет собой возможность получать консультации специалистов компании-разработчика программного продукта.

Программный продукт может продаваться под отдельной коммерческой лицензией или распространяться бесплатно вместе с исходным кодом под свободной лицензией (хотя двоичные сборки свободного программного продукта могут быть формально не бесплатны, наличие исходного кода в открытом доступе позволяет получить из него исполняемый двоичный файл, идентичный коммерческой версии \cite{oracleMovesToRedhat}). Программное обеспечение в таком случае — разовая покупка или «бесплатное приложение» к труду людей, оплачиваемому на регулярной основе, повод купить повременной труд технических специалистов. Как и в случае с SaaS, это сокращение необходимых издержек на обслуживание ИТ-инфраструктуры. Однако, в отличие от модели SaaS, аппаратная инфраструктура остаётся внутри компании, а экономия достигается за счет повышения производительности труда администраторов системы, которые тратят меньше времени на решение проблем и реализацию специфических задач, получая консультации внешних оплаченных по подписке специалистов\footnote{«Finally, Red Hat Services has certified consultants available to accelerate your work and reduce time to value. These services can only be used in the context of a paid subscription.» \cite{redhatSubscriptionGuide}}.

Модель продажи поддержки на ПО распространена среди коммерческих компаний, распространяющих программные продукты вместе с открытым исходным кодом под свободной лицензией. При этом компания может являться непосредственным разработчиком программного продукта, вносить больший или меньший вклад в его разработку наряду с другими игроками или не вносить вклад в разработку, но предоставлять услуги по его обслуживанию и эксплуатации. В том случае, если компания несет издержки по разработке программного продукта, она их компенсирует из дохода от продажи поддержки.

\section*{Так называемые «интернет-сервисы»: Гугл, Яндекс, Ютюб, Твиттер, Телеграм и т.~п.}

Технически интернет-порталы, нацеленные на массовую аудиторию частных пользователей, такие, как поисковые системы, социальные сети, мессенджеры, хостинги видео, хостинги изображений, сервисы бесплатной электронной почты и т.~п. представляют собой серверную (облачную) аппаратно-программную инфраструктуру с доступом через канал интернет через веб-интерфейс или специализированное приложение, т.~е. по внешним признакам и технологиям реализации очень похожи на модель SaaS.

Однако, в отличие от модели SaaS, возможности таких интернет-порталов доступны для массового пользователя бесплатно, между пользователем и порталом не возникает отношения купли-продажи. Пользовательское соглашение обычно включает обязательства пользователя соблюдать правила портала, уведомление о том, что владелец портала может ограничить доступ пользователя к сервису в любой момент без объяснения причин, но не обязательства портала перед пользователем. Таким образом, хотя частные пользователи интернет-порталов используют их программное обеспечение, аппаратные ресурсы и прочую серверную инфраструктуру для своих целей, нельзя говорить о том, что интернет-портал оказывает пользователю услугу (предоставляет сервис) — по крайней мере, с позиции экономических отношений.

Основная выручка крупнейших массовых интернет-порталов приходит по каналам рекламы. К примеру, выручка компании Фейсбук (в настоящее время — Мета, запрещена в РФ) в 2019 году составила \$70.697 млрд, из них от рекламы — \$69.655 млрд, чистый доход — \$18,485 млрд. \cite{facebook10K2019}. Таким образом, экономические отношения у массовых интернет-порталов возникают с рекламодателями — организациями, предлагающими пользователям портала через рекламные механизмы площадки свой продукт. Аппаратная серверная инфраструктура, энергия, канал связи, обслуживающий персонал, прочие операционные издержки текущей деятельности, а также первоначальные издержки на разработку программного обеспечения интернет-портала оплачены и компенсированы из средств рекламодателей, они же и являются его клиентами, т.~е. получателями услуги. Таким образом, массовые интернет-порталы (т.~н. интернет-сервисы) — это агрегированные по обществу непроизводственные издержки на рекламу.

С точки зрения рекламодателя выгода от использования интернет-портала для размещения рекламы — сокращение собственных необходимых издержек на рекламу. Если купить рекламу у интернет-портала окажется дешевле, чем содержать, к примеру, собственный отдел продаж, а отдача от такой рекламы окажется такая же или выше, то это будет объективным экономическим стимулом совершить покупку рекламы. Если же расходы на рекламу на выбранном портале в пересчете на реализованный продукт окажутся выше, чем такой же показатель для другого механизма обеспечения сбыта, покупать рекламу на таком портале будет экономически нецелесообразно.

Определение универсальной формулы для расчета стоимости размещения одного рекламного блока представляется крайне сложной, если вообще разрешимой задачей, т.~к. формы и механизмы размещения рекламы меняются со временем и от портала к порталу, в некоторых случаях они будут не вполне тривиальны. Но общий принцип заключается в том, что суммарная стоимость размещения от всех рекламодателей в выбранный период должна покрыть работу интернет-сервиса как аппаратно-программного комплекса — амортизацию оборудования, энергию, канал связи, работу обслуживающего персонала, прочие операционные расходы, — за этот же период времени. Первоначальные издержки на разработку ПО портала и прочие инвестиции будут компенсированы со временем из прибыли от текущей деятельности.

Наличие активной аудитории интернет-портала — необходимое условие для того, чтобы он имел возможность продавать рекламу. Для того, чтобы отдача от рекламы на портале была приемлемого уровня при приемлемой цене размещения, аудитория интернет-портала должна быть достаточно широка. В современном мире — это масштаб крупного государства или доли мирового рынка. После того, как портал уже запущен, он должен набрать достаточное количество аудитории — пройти период роста. К этому времени основные первоначальные издержки на разработку базового ПО и закупку аппаратной платформы уже понесены. В течение этого периода портал функционирует в полноценном режиме — потребляет энергию, связь, труд обслуживающего персонала, увеличение нагрузки может потребовать закупки дополнительного аппаратного обеспечения и расширения штата сотрудников. Но до тех пор, пока масса аудитории не достигла некоторого критического уровня, канал компенсации текущих расходов через продажу рекламы не открыт. Это дополнительные издержки, которые несут инвесторы проекта до тех пор, пока не начнет функционировать бизнес-модель — до выхода на окупаемость проект функционирует на средства инвесторов. Кроме расходов на текущую и расширяющуюся деятельность, привлечение аудитории требует значительных расходов на рекламу (рекламу самого сервиса на сторонних площадках). Период роста может занимать продолжительное время: обычная практика — до нескольких лет. В течение этого времени более важными показателями для инвесторов могут быть динамика роста и вовлечённость аудитории, чем текущие финансовые показатели такие, как выручка или прибыль.

Особенность порталов, дающих возможность пользователям создавать или загружать в публичный доступ контент, — необходимость модерировать контент, как минимум для того, чтобы он не нарушал действующее законодательство. Поэтому кроме системных администраторов облачную инфраструктуру на постоянной основе обслуживает команда контент-модераторов \cite{facebookModerators}, стоимость рабочей силы которых присоединяется к стоимости издержек на содержание сервиса.

\section*{Смешанные модели: реклама vs SaaS}

Многие массовые публичные интернет-сервисы, к примеру, Ютюб, Телеграм или Живой Журнал, получающие доход преимущественно за счет продажи рекламы, предоставляют своим пользователям возможность покупки платного (премиального) аккаунта. Некоторые сервисы, работающие по модели SaaS, получающие доход преимущественно за счет продажи аппаратных ресурсов, предоставляют своим пользователям возможность бессрочно работать с бесплатных аккаунтов, которые, однако, подразумевают показы рекламы (например: Яндекс Дискs). В рамках одного и того же сервиса часть пользователей, заплатившая за премиальные аккаунты, таким образом напрямую оплачивает потребляемые аппаратные ресурсы и прочую инфраструктуру. Стоимость оставшейся доли аппаратных ресурсов, потребляемой пользователями бесплатных аккаунтов, компенсируется рекламодателями по модели продажи рекламы. Обычная опция премиальных аккаунтов — отключение рекламы в платном режиме \cite{youtubePremium}.

\section*{Портал Госуслуги}

Портал Государственные услуги, как и прочие порталы взаимодействия граждан и специальных государственных учреждений и министерств (ФНС, ГИБДД), сайты муниципалитетов и т.~п., технически представляет собой аппаратно-программную серверную инфраструктуру, аналогичную продуктам SaaS и порталам «бесплатных» массовых интернет-сервисов. В отличие от модели SaaS, сам потребитель прямо не оплачивает доступ к порталу Госуслуг. В отличие от модели массовых интернет-сервисов, портал Госуслуги не использует модель продажи рекламы для оплаты стоимости серверной инфраструктуры.

Функционирующий портал Госуслуг экономит время граждан, а также повышает производительность труда государственных чиновников, это экономит государственный бюджет, в этом заключается его прямой экономический эффект. Средства на его разработку и содержание выделяет общество в целом — через государственный бюджет, они составляют непроизводственные издержки в масштабах общества. Государственный бюджет покрывает как изначальные издержки на разработку ПО, так и текущие расходы на закупку, развертывание и содержание аппаратно-программной платформы и всей сопутствующей инфраструктуры портала с обслуживающим персоналом.

\section*{Частное потребление (b2c): игры}

В настоящее время игровые программные продукты представлены на основных платформах: ПК, консоли, мобильные гаджеты, веб, аппаратно-программные платформы (игровые автоматы); и могут работать в рамках основных моделей: оффлайн-лицензия, оффлайн-подписка, SaaS, реклама, амортизация специализированного устройства (коллективное потребление игрового автомата).

Традиционная платформа для оффлайн-лицензий — ПК и консоли, с развитием рынка смартфонов и планшетов — мобильные гаджеты. Распространение специализированных магазинов приложений (маркетплейсов) на этих платформах облегчило процедуру покупки и доставки приложения на устройство пользователя, а также контроль за использованием лицензии для оффлайн-лицензии и оффлайн-подписки. По модели SaaS работают массовые многопользовательские онлайн игры, оплата серверной инфраструктуры может осуществляться как напрямую через платные аккаунты, так и косвенно — через «покупку» внутриигровых предметов. На мобильных гаджетах и веб-платформах для игр распространена смешанная модель: для «бесплатных» установок — реклама, альтернатива — покупка бессрочной или срочной лицензии, которая рекламу скрывает. Неделимой аппаратно-программной платформой, потребляемой коллективно, но в режиме оффлайн, являются игровые автоматы.

Особенность игрового сегмента заключается в том, что он нацелен в первую очередь на частное потребление, а не на бизнес. Игры не повышают производительность труда на производстве, не сокращают непроизводственные издержки, поэтому речи об экономических эффектах автоматизации здесь не идет. Источник ресурсов — кошелек частного потребителя. Стимул для покупки — субъективный пользовательский опыт, включающий впечатления от визуальной составляющей, игрового процесса, сюжета и т.~п. Важную роль играет мнение других игроков, масштаб и качество рекламной кампании.

Значительная часть предварительных издержек, кроме непосредственно разработки игры, — реклама перед запуском. Стоимость разработки дополнения «Phantom Liberty» к игре Cyberpunk 2077 составила \$62.8 миллионов при рекламном бюджете \$21.7 миллионов \cite{cyberpunkMarketingBudget}. Общий бюджет базовой версии игры Cyberpunk 2077 составил примерно \$316 миллионов \cite{cyberpunkTotalBudget}, включая расходы на разработку и на рекламу. Хотя компания-разработчик CD Projekt не раскрыла соотношение расходов на разработку и рекламу для базовой версии игры, с учетом того, что рекламная кампания перед первоначальным запуском проходила в более интенсивном режиме, можно предположить, что в этом случае перевес был в большей степени в сторону рекламы (по одной из грубых оценок стоимость разработки могла составить порядка \$100 миллионов \cite{cyberpunkDevBudgetEstimate}.

Конкуренты — другие игры (в аналогичном жанре или любом другом жанре, т.~к. вкусовые предпочтения одного потребителя могут варьироваться), любые другие продукты потребления и способы провождения свободного времени (выбор: «поиграть», «посмотреть фильм», «почитать книгу», «сходить в спортивный зал», «погулять», «просто отдохнуть»), инфляция, проценты по кредиту, стоимость услуг ЖКХ и т.~п.

Процесс разработки игры, кроме работы над программной составляющей, обычно включает необходимость создания игровых ресурсов — моделей, текстур, дизайна и наполнения уровней, музыки, звуковых эффектов, озвучки персонажей, прочего мультимедийного контента. Для больших проектов эта часть разработки может оказаться наиболее существенной по затратам. В части программной составляющей многие проекты используют готовые специализированные программные модули, фреймворки или платформы — т.~н. игровые «движки». Разработчики игровых движков могут распространять продукт среди разработчиков игр по разным моделям: например, для малых некоммерческих или учебных проектов предоставлять движок бесплатно, для коммерческих продуктов на стадии эксплуатации и продаж — продавать лицензию фиксированной стоимости или на условиях получения процента от продаж, а также дополнительно продавать к нему поддержку или включать её в цену лицензии.

Хотя финальная цена единичной копии игрового программного продукта не ограничена экономическим эффектом автоматизации, у частных потребителей могут сложиться определенные ценовые ожидания от продуктов разного уровня качества, попадающих в разные ценовые диапазоны (проекты уровня ААА — топовая графика, максимальное использование возможностей современного «железа»; средний ценовой диапазон, «независимые» «инди»-проекты, казуальные игры и т.~п.). Размер потенциальной аудитории можно оценить по данным о количестве пользователей целевых программных и аппаратно-программных платформ — десктоп, игровые приставки, мобильные устройства. Чем дороже разработка проекта, тем большую аудиторию потребуется привлечь, чтобы попасть в целевой ценовой диапазон и окупить издержки на разработку проекта и рекламу \cite{cyberpunkProfit}, тем больше ресурсов необходимо вложить в первоначальную рекламную кампанию.

\section*{Прямое перераспределение стоимости с ИТ}

Камеры дорожного наблюдения, прочие автоматические штрафы. Система камер дорожного наблюдения будет включать основные модули: распознавание объектов на дороге, дорожных ситуаций, знака авто (ИИ — специализированный программный алгоритм), система автоматического выписывания штрафов (интернет, ИТ-инфраструктура — база данных, программная логика приложения), аппаратная платформа (облачная инфраструктура, сеть уличных камер и т.~п.), обслуживающий персонал — системные администраторы и пр..

Так же, как в случае с частным потреблением программных продуктов, источник ресурсов для компенсации исходных издержек на разработку и развертывание аппаратно-программной платформы и текущих операционных расходов — средства частных лиц \cite{autoFineRF}. Но, в отличие от игр и прочих программных продуктов, предназначенных для частного потребления, в рамках этой модели не требуется нести издержки на рекламу.

\section*{Добровольные пожертвования}

В том случае, если разработчик программного продукта распространяет его бесплатно, в том числе с исходным кодом под свободной лицензией, при этом не применяет к нему рассмотренные или иные схемы коммерциализации, он все еще может полностью или частично компенсировать собственные издержки на разработку, принимая добровольные безвозмездные пожертвования без возникновения прямых обязательств перед отправляющей пожертвование стороной. Принимающая пожертвование сторона обычно — некоммерческие фонды разработки свободного программного обеспечения (например: GNU FSF, Linux Foundation, Apache Software Foundation) или индивидуальные разработчики, получающие денежные переводы напрямую или через специализированную платформу (например: Patreon). Источник средств — частные спонсоры («донаты») или корпоративные спонсоры. Пожертвования могут полностью или частично компенсировать издержки на разработку программного продукта или компенсировать сопутствующие расходы на инфраструктуру или деятельность, связанные с проектом — оплата хостинга, проведение конференций и т.~п.

Если оставить мотивы, не связанные напрямую с рассматриваемым программным продуктом, такие как альтруизм (со стороны частных спонсоров), получение налоговых вычетов или вложения в репутацию, т.~е. проведение пожертвования по линии рекламного бюджета, (со стороны корпоративных спонсоров) в стороне, объективным мотивом для пожертвования может быть желание получить новую улучшенную версию программного продукта. Пожертвование может быть направлено на развитие проекта «в целом», или же целевым образом направлено на решение конкретной проблемы — исправление специфической ошибки или добавление определенной новой функции. В том случае, если необходимые новшества будут реализованы разработчиком проекта и войдут в очередной релиз программного продукта, спонсор проекта может частично или полностью окупить направленные им средства через реализацию экономического эффекта от внедрения в собственный производственный процесс улучшенной версии программного пакета. Т.~к. улучшенная версия продукта выпускается свободно в открытый доступ, экономический эффект в масштабах общества будет шире, чем частная выгода отправивших пожертвования спонсоров, — его смогут реализовать любые игроки, использующие программный продукт, в том числе те, кто не отправлял в проект пожертвования.

\section*{Смешанные модели: ОС Google Android}

Программная часть для аппаратно-программной платформы — мобильных устройств — смартфонов и планшетов. Распространяется как свободное программное обеспечение. Значительная часть кодовой базы берется из общедоступного «фонда» свободного программного обеспечения (ядро Linux, программные библиотеки и т.~п.). Другая значительная часть — разработка корпорации Google, также публикуемая как свободное программное обеспечение.
 Производители устройств или типовых материнских плат берут базовую программную часть платформы как есть или вносят дополнительные доработки (добавляют закрытые драйверы устройств, брендируют под себя интерфейс), используют её в качестве базовой операционной системы собственных устройств. Таким образом понижается потолок выхода на рынок нового устройства, по крайней мере, на него в меньшей степени оказывает влияние фактор необходимых издержек на разработку или лицензирование базовой программной прошивки.

Многие производители устройств вместе с операционной системой ставят на устройства пакет программ — сервисы Google. Google получает новых пользователей — покупателей смартфонов, привязанных к её бесплатным «сервисам» (поиск, почта, карты и т.~п.), замыкая на себя значительную долю мирового мобильного интернет-трафика.: о Основной источник дохода компании — реклама. Таким образом, издержки на разработку бесплатной открытой операционной системы Google компенсирует из дохода от рекламы, который, в свою очередь, сохраняется и увеличивается благодаря тому, что мобильная операционная система, разрабатываемая компанией Google, доминирует на мобильном рынке; доминирование, в свою очередь, в значительной степени обеспечено тем, что значительная доля необходимого программного обеспечения доступна производителям устройств бесплатно.

% чтобы ссылки в тексте имели правильные номера, генератор нужно запустить два раза
% нижние подчеркивания '_' в ссылках заменяем на '\_', иначе они воспринимаются как спец-символы
\begin{thebibliography}{2}

\bibitem{ecoEffects} Моисеев, А.~Е. Оценка эффективности цифровой трансформации секторов экономики / А.~Е.~Моисеев, Н.~А.~Мурашова // Инновации и инвестиции. – 2023. – № 7. – С. 388-391. – EDN IXZPSY.
\bibitem{phoneUpdates} Here are the phone update policies from every major Android manufacturer / C.~Scott Brown // Android Authority, – 2023. [Электронный ресурс]. – URL: https://www.androidauthority.com/phone-update-policies-1658633/
\bibitem{microsoftPiracyReinvest} Q\&A: How Software Piracy Undermines Economic Recovery / Nancy Anderson // PressPass, – 2001. [Электронный ресурс]. – URL: https://news.microsoft.com/2001/10/19/qa-how-software-piracy-undermines-economic-recovery/
\bibitem{accountingIAS} Международный стандарт финансовой отчетности (IAS) 38 «Нематериальные активы» // Минфин России, – 2023. [Электронный ресурс]. – URL: https://minfin.gov.ru/ru/document/?id\_4=15329
\bibitem{accountingNKRF257} НК РФ Статья 257. Порядок определения стоимости амортизируемого имущества. П.3.: нематериальные активы. // Налоговый кодекс РФ, – 2023. [Электронный ресурс]. – URL: https://www.consultant.ru/document/cons\_doc\_LAW\_28165/cf1a9426ba878faee9824672bca283c1420a2b1e/
\bibitem{accountingNKRF264} НК РФ Статья 264. Прочие расходы, связанные с производством и (или) реализацией. П.26.: расходы, связанные с приобретением права на использование программ для ЭВМ и баз данных по договорам с правообладателем (по лицензионным и сублицензионным соглашениям) // Налоговый кодекс РФ, – 2023. [Электронный ресурс]. – URL: https://www.consultant.ru/document/cons\_doc\_LAW\_28165/3fdee9a04c76f1af1e084502759523cd77da7d16/
\bibitem{driscollOpenLetter} Driscoll, Kevin. "Professional Work for Nothing: Software Commercialization and “An Open Letter to Hobbyists”." Information \& Culture: A Journal of History, vol. 50 no. 2, 2015, p. 257-283. Project MUSE, https://doi.org/10.1353/lac.2015.0005
Introduction to Digital Economics: Foundations, Business Models and Case Studies / Harald Øverby, Jan Arild Audestad // Springer Cham. – 2021. – С. 78-80. – DOI 10.1007/978-3-030-78237-5. – eBook ISBN 978-3-030-78237-5.
\bibitem{introToDigital} Introduction to Digital Economics: Foundations, Business Models and Case Studies / Harald Øverby, Jan Arild Audestad // Springer Cham. – 2021. – С. 78-80. – DOI 10.1007/978-3-030-78237-5. – eBook ISBN 978-3-030-78237-5.
\bibitem{windowsXPLegacy} Microsoft begs users to leave Windows XP – gives them until 2014 to do it / Jeffrey Van Camp // Digital Trends, – 2011. [Электронный ресурс]. – URL: https://www.digitaltrends.com/computing/microsoft-begs-users-to-leave-windows-xp-gives-them-until-2014-to-do-it/
\bibitem{autodeskSubscription} Using Autodesk subscription products without an internet connection // Autodesk, – 2023. [Электронный ресурс]. – URL: https://www.autodesk.com/support/technical/article/caas/sfdcarticles/sfdcarticles/About-using-a-Autodesk-product-on-subscription-without-an-Internet-connection.html
\bibitem{microsoftInvestsRF} За три года Microsoft инвестирует в Россию 10 млрд руб. / Татьяна Короткова // CNews, – 2009. [Электронный ресурс]. – URL: https://www.cnews.ru/news/line/za\_tri\_goda\_microsoft\_investiruet\_v
\bibitem{cloudIsAFancyName} Disrupting Nation State Hackers / Rob Joyce // USENIX Association, – 2016. [Электронный ресурс]. – URL: https://www.usenix.org/conference/enigma2016/conference-program/presentation/joyce
\bibitem{microsoftSaasStrategy} Microsoft’s SaaS Strategy: A Giant Copes with Change / James Maguire // Datamation, – 2007. [Электронный ресурс]. – URL: https://www.datamation.com/networks/microsofts-saas-strategy-a-giant-copes-with-change/
\bibitem{microsoftMoveToSubscriptionAndSaas} Microsoft's move to Subscription and SaaS / David Foxen // ITAM Review, – 2015. [Электронный ресурс]. – URL: https://itassetmanagement.net/2015/06/02/microsofts-move-subscription-saas/
\bibitem{erpSaas} The future of business ERP is SaaS. Reporting from Directions EMEA 2021 // LS Retail, – 2021. [Электронный ресурс]. – URL: https://www.lsretail.com/resources/the-future-of-business-erp-is-saas-reporting-from-directions-emea-2021
\bibitem{sellingServices} How to Shift from Selling Products to Selling Services / Doug J.~Chung // Harvard Business Review, – 2021. [Электронный ресурс]. – URL: https://hbr.org/2021/03/how-to-shift-from-selling-products-to-selling-services
\bibitem{dropboxBuildsOwnInfra} Dropbox saved almost \$75 million over two years by building its own tech infrastructure / Tom Krazit // GeekWire, – 2018. [Электронный ресурс]. – URL: https://www.geekwire.com/2018/dropbox-saved-almost-75-million-two-years-building-tech-infrastructure/
\bibitem{redhatSubscriptionModel} Red Hat subscription model FAQ: What's included in a subscription? // Red Hat, – 2023. [Электронный ресурс]. – URL: https://www.redhat.com/en/about/subscription-model-faq
\bibitem{oracleMovesToRedhat} Oracle makes its move on Red Hat / ITP Staff // edge, – 2006. [Электронный ресурс]. – URL: https://www.edgemiddleeast.com/news/488113-oracle-makes-its-move-on-red-hat
\bibitem{redhatSubscriptionGuide} Red Hat Enterprise Linux subscription guide // Red Hat, – 2023. [Электронный ресурс]. – URL: https://www.redhat.com/en/resources/red-hat-enterprise-linux-subscription-guide
\bibitem{facebook10K2019} Facebook, Inc., Form 10-K: Annual report for the fiscal year ended December 31, 2019 // U.S. Securities and Exchange Commission, – 2020. [Электронный ресурс]. – URL: https://www.sec.gov/ix?doc=/Archives/edgar/data/1326801/000132680120000013/fb-12312019x10k.htm
\bibitem{facebookModerators} Facebook moderators: a quick guide to their job and its challenges / Nick Hopkins // The Guardian, – 2017. [Электронный ресурс]. – URL: https://www.theguardian.com/news/2017/may/21/facebook-moderators-quick-guide-job-challenges
\bibitem{youtubePremium} Какие возможности дает YouTube Premium // Google, – 2023. [Электронный ресурс]. – URL: https://support.google.com/youtube/answer/6308116
\bibitem{cyberpunkMarketingBudget} Cyberpunk 2077: Phantom Liberty Had a Development and Marketing Budget of Nearly \$85 Million / Shubhankar Parijat // GamingBolt, – 2023. [Электронный ресурс]. – URL: https://gamingbolt.com/cyberpunk-2077-phantom-liberty-had-a-development-and-marketing-budget-of-nearly-85-million
\bibitem{cyberpunkTotalBudget} Cyberpunk 2077’s Total Budget Was Roughly \$316 Million / Ravi Sinha // GamingBolt, – 2021. [Электронный ресурс]. – URL: https://gamingbolt.com/cyberpunk-2077s-total-budget-was-roughly-316-million
\bibitem{cyberpunkDevBudgetEstimate} Cyberpunk 2077 Budget: How much did it cost to make the game? / Hrithik Raj // Sportskeeda, – 2020. [Электронный ресурс]. – URL: https://www.sportskeeda.com/esports/cyberpunk-2077-budget-how-much-cost-make-game
\bibitem{cyberpunkProfit} After eight years of development, Cyberpunk 2077 made a profit in one day /Andy Chalk // PC Gamer, – 2020. [Электронный ресурс]. – URL: https://www.pcgamer.com/after-eight-years-of-development-cyberpunk-2077-made-a-profit-in-one-day/
\bibitem{autoFineRF} Куда и на что уходят автомобильные штрафы / Майя Бирюкова // Российская газета, – 2020. [Электронный ресурс]. – URL: https://rg.ru/2020/09/03/kuda-i-na-chto-uhodiat-avtomobilnye-shtrafy.html

\end{thebibliography} 

\end{document}

